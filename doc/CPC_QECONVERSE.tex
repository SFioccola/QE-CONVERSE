%% This template can be used to write a paper for
%% Computer Physics Communications using LaTeX.
%% For authors who want to write a computer program description,
%% an example Program Summary is included that only has to be
%% completed and which will give the correct layout in the
%% preprint and the journal.
%% The `elsarticle' style is used and more information on this style
%% can be found at 
%% http://www.elsevier.com/wps/find/authorsview.authors/elsarticle.
%%
%%

\documentclass[final,3p,times,twocolumn]{elsarticle}
%\documentclass[preprint,12pt]{elsarticle}

%% Use the option review to obtain double line spacing
%% \documentclass[preprint,review,12pt]{elsarticle}

%% Use the options 1p,twocolumn; 3p; 3p,twocolumn; 5p; or 5p,twocolumn
%% for a journal layout:
%% \documentclass[final,1p,times]{elsarticle}
%% \documentclass[final,1p,times,twocolumn]{elsarticle}
%% \documentclass[final,3p,times]{elsarticle}
%% \documentclass[final,3p,times,twocolumn]{elsarticle}
%% \documentclass[final,5p,times]{elsarticle}
%% \documentclass[final,5p,times,twocolumn]{elsarticle}

%% if you use PostScript figures in your article
%% use the graphics package for simple commands
%% \usepackage{graphics}
%% or use the graphicx package for more complicated commands
%% \usepackage{graphicx}
%% or use the epsfig package if you prefer to use the old commands
%% \usepackage{epsfig}

%% The amssymb package provides various useful mathematical symbols
\usepackage[colorlinks,allcolors=blue]{hyperref}
\usepackage{amssymb}
\usepackage[english]{babel}
\usepackage{amsmath}
\usepackage{tikz}
\usepackage{amsmath} % Per le equazioni
\usetikzlibrary{shapes.geometric, arrows}
\usepackage{caption}
\usepackage{threeparttable}
%% The amsthm package provides extended theorem environments
%% \usepackage{amsthm}

%% The lineno packages adds line numbers. Start line numbering with
%% \begin{linenumbers}, end it with \end{linenumbers}. Or switch it on
%% for the whole article with \linenumbers after \end{frontmatter}.
%% \usepackage{lineno}

%% natbib.sty is loaded by default. However, natbib options can be
%% provided with \biboptions{...} command. Following options are
%% valid:

%%   round  -  round parentheses are used (default)
%%   square -  square brackets are used   [option]
%%   curly  -  curly braces are used      {option}
%%   angle  -  angle brackets are used    <option>
%%   semicolon  -  multiple citations separated by semi-colon
%%   colon  - same as semicolon, an earlier confusion
%%   comma  -  separated by comma
%%   numbers-  selects numerical citations
%%   super  -  numerical citations as superscripts
%%   sort   -  sorts multiple citations according to order in ref. list
%%   sort&compress   -  like sort, but also compresses numerical citations
%%   compress - compresses without sorting
%%
%% \biboptions{comma,round}

% \biboptions{}

%% This list environment is used for the references in the
%% Program Summary
%%
\newcounter{bla}
\newenvironment{refnummer}{%
\list{[\arabic{bla}]}%
{\usecounter{bla}%
 \setlength{\itemindent}{0pt}%
 \setlength{\topsep}{0pt}%
 \setlength{\itemsep}{0pt}%
 \setlength{\labelsep}{2pt}%
 \setlength{\listparindent}{0pt}%
 \settowidth{\labelwidth}{[9]}%
 \setlength{\leftmargin}{\labelwidth}%
 \addtolength{\leftmargin}{\labelsep}%
 \setlength{\rightmargin}{0pt}}}
 {\endlist}
\usepackage{url}
\usepackage{multirow}
\journal{Computer Physics Communications}

\begin{document}

\begin{frontmatter}

%% Title, authors and addresses

%% use the tnoteref command within \title for footnotes;
%% use the tnotetext command for the associated footnote;
%% use the fnref command within \author or \address for footnotes;
%% use the fntext command for the associated footnote;
%% use the corref command within \author for corresponding author footnotes;
%% use the cortext command for the associated footnote;
%% use the ead command for the email address,
%% and the form \ead[url] for the home page:
%%
%% \title{Title\tnoteref{label1}}
%% \tnotetext[label1]{}
%% \author{Name\corref{cor1}\fnref{label2}}
%% \ead{email address}
%% \ead[url]{home page}
%% \fntext[label2]{}
%% \cortext[cor1]{}
%% \address{Address\fnref{label3}}
%% \fntext[label3]{}

\title{{\fontfamily{qcr}\selectfont
QE-CONVERSE}: { An open-source code for Quantum ESPRESSO package to compute Orbital Magnetization by first principles }}

%% use optional labels to link authors explicitly to addresses:
%% \author[label1,label2]{<author name>}
%% \address[label1]{<address>}
%% \address[label2]{<address>}

\author[a]{S. Fioccola\corref{author}}
\author[b]{L. Giacomazzi}
\author[c]{D. Ceresoli}
\author[d]{N. Richard}
\author[a]{A. Hemeryck}
\author[b]{L. Martin-Samos}


\cortext[author] {Corresponding author.\\\textit{E-mail address:} sfioccola@laas.fr}
\address[a]{LAAS-CNRS, Université de Toulouse, CNRS, Toulouse, France;}
\address[b]{CNR-IOM, Istituto Officina dei Materiali, National Research Council of Italy, c/o SISSA, via Bonomea 265, Trieste, Italy}
\address[c]{CNR-SCITEC – Istituto di Scienze e Tecnologie Chimiche “G. Natta”, National Research Council of Italy, via C. Golgi 19, Milano 20133, Italy}
\address[d]{CEA, DAM, DIF, Arpajon, France}


\begin{abstract}
%% Text of abstract
In this paper, we present a revised implementation of the converse method for computing orbital magnetization from first principles. This new version, developed for the Quantum ESPRESSO package, replaces the outdated previous version and makes extensive use of modern libraries for linear algebra computation, including scaLAPACK and ELPA, enhancing the efficiency of calculations, especially in the context of supercells. Orbital magnetization calculation  holds significant importance as it allows for the ab-initio calculation of the EPR g tensor and NMR chemical shift in a non-perturbative manner, providing valuable insights into the magnetic properties of materials.  Our {\fontfamily{qcr}\selectfont
QE-CONVERSE} code is well-suited for investigating these properties in periodic systems.

\end{abstract}

\begin{keyword}
%% keywords here, in the form: keyword \sep keyword
EPR, NMR, Ab-initio, Orbital Magnetization, Quantum ESPRESSO

\end{keyword}

\end{frontmatter}

%%
%% Start line numbering here if you want
%%
% \linenumbers

% All CPiP articles must contain the following
% PROGRAM SUMMARY.

{\bf Program summary}
  %Delete as appropriate.

\begin{small}
\noindent
{\em Program Title:} qe-converse                                       \\
{\em CPC Library link to program files:} (to be added by Technical Editor) \\
{\em Developer's repository link:} \url{https://github.com/SFioccola/QE-CONVERSE} \\
{\em Code Ocean capsule:} (to be added by Technical Editor)\\
{\em Licensing provisions(please choose one):} GNU General Public Licence 3.0  \\
{\em Programming language:}     Fortran 90                              \\
{\em Supplementary material:}                                 \\
  % Fill in if necessary, otherwise leave out.
{\em Journal reference of previous version:} \\ \url{https://doi.org/10.1103/PhysRevB.81.060409} \\
\url{https://doi.org/10.1103/PhysRevB.81.184424}\\
  %Only required for a New Version summary, otherwise leave out.
{\em Does the new version supersede the previous version?:} yes   \\
  %Only required for a New Version summary, otherwise leave out.
{\em Reasons for the new version:} The previous code was integrated into an obsolete version of the PW code in the Quantum ESPRESSO package. The new version is compatible with the modern package and utilizes modern libraries for linear algebra computation.  \\
  %Only required for a New Version summary, otherwise leave out.
{\em Summary of revisions:} The code underwent a process of refactoring. \\
  %Only required for a New Version summary, otherwise leave out.
{\em Nature of problem:} Ab-initio calculation of the EPR g tensor and the NMR chemical shift in solid state.\\
  %Describe the nature of the problem here. \\
{\em Solution method:} Compute the Orbital Magnetization through a non-pertubative method. \\
  %Describe the method solution here.
{\em Additional comments including restrictions and unusual features (approx. 50-250 words):}\\
  %Provide any additional comments here.
   \\
\paragraph{1. Introduction} \ 
\vskip0.5cm
\noindent
Orbital magnetization, a fundamental property arising from the orbital motion of electrons, plays a crucial role in determining the magnetic behavior of molecules and solids. Alongside spin magnetization, it contributes significantly to the total magnetization observed in various materials. The origin of orbital magnetization is connected in the breaking of time-reversal (TR) symmetry, which is induced by spin-orbit (SO) coupling. This phenomenon occurs spontaneously in ferromagnetic materials or can be triggered in nonmagnetic materials by explicit perturbations.
The calculation of orbital magnetization allows ab-initio computation of macroscopic properties like the Nuclear Magnetic Resonance (NMR) chemical shifts and the Electronic Paramagnetic Resonance (EPR) g tensor through a non-perturbative approach known as the "converse" method. This method can be highly superior to existing and widely spread used linear response (LR) method \cite{PhysRevLett.88.086403}, especially in systems where spin-orbit coupling cannot be described as perturbation or the g tensor exhibits a large deviation from the free electron value \cite{PhysRevB.81.060409,PhysRevB.81.195208}.
While the calculation of the orbital magnetization in finite systems is straightforward, it becomes challenging in periodic systems due to the ill-defined position operator in Bloch representations and the contribution of itinerant surface current \cite{Resta_2010,PhysRevB.74.024408}. Over the past two decades, this challenge has been addressed in the modern theory of orbital magnetization \cite{PhysRevLett.95.137204,PhysRevLett.95.137205} by reformulating the problem in the Wannier representation, similar to the approach used for electric polarization. This reformulation enables the computation of orbital magnetization in the thermodynamic limit, employing a formula reminiscent of the Berry-phase formula. \\  Recently, the converse approach has been implemented \cite{PhysRevB.81.060409} in the Quantum ESPRESSO \cite{Giannozzi2017} plane wave code, extending the ab-initio calculation of orbital magnetization via the Berry formula \cite{PhysRevB.74.024408} from simple tight-binding models to molecules and periodic systems. In this paper, we present a novel implementation of the converse method that, building upon the foundation of the previous version, continues to rely on first-principle DFT calculations of orbital magnetization using the Berry phase formula. However, a notable enhancement lies in the integration of modern linear algebra libraries, including scaLAPACK \cite{slug} and ELPA \cite{Marek_2014}. 
These advanced libraries significantly improve the efficiency of matrix diagonalization processes especially for larger-scale systems, where computational resources are often a limiting factor. By leveraging these libraries, our upgraded approach enables researchers to tackle the computation of properties such as the EPR g tensor or the NMR chemical shift in larger and more complex systems with improved efficiency and accuracy. This enhancement represents a significant step forward in the practical utility of ab-initio calculations for studying the magnetic properties of materials. This paper is organized as follows. In Section 2, we briefly review the theoretical background of converse method. In Section 3 and 4 we explain the implementation, the installation and usage of {\fontfamily{qcr}\selectfont
QE-CONVERSE} code. Finally we provide a couple of applications: one concerning the EPR g tensor calculation for the substitutional Nitrogen defect in Silicon (N\textsubscript{Si}), and another one concerning the NMR chemical shift of \ ${}^{27}$Al\ in alumina ($\mathrm{Al_{2}O_{3}}$). \\
\vskip0.5cm
%%Theoretical Background 
\paragraph{2. Theoretical background} \
\vskip0.5cm
\noindent
\subparagraph{2.1. Converse EPR}\ \\ \\
The converse method for calculating the EPR g tensor has been introduced in Ref.\cite{PhysRevB.81.060409} and can be summarized as follows. The basic starting point is an independent particles Kohn-Sham Hamiltonian having the translation symmetry of the crystal but breaking Time-Reversal (TR) symmetry. Within the density-functional theory (DFT) and the all-electron (AE) formalism, the Hamiltonian in atomic units is: \\ \\
\begin{equation}
H_{\rm{AE}}=\frac{1}{2}[\mathbf{p}+\alpha A(\mathbf{r})]^{2}+V(\mathbf{r})+\frac{\alpha^{2}g'}{8}\sigma\cdot[\nabla V(\mathbf{r})\times \mathbf{p}+\alpha A(\mathbf{r})]
\end{equation} 
where $A(\mathbf{r})$ is the symmetric gauge equal to $\frac{1}{2} \mathbf{B}\times\mathbf{r}$, $\alpha=\frac{1}{c}$ is the fine structure constant, $g'=2(g_{e}-1)$  and $\sigma$ are Pauli matrices. In contrast to the linear response approach, the converse method bypasses the perturbation of the Hamiltonian: the spin-orbit coupling (SO) term is explicitly incorporated into the self-consistent field calculations while the spin other orbit (SOO) term, that in general has a small contribution to the g tensor, is neglected. The orbital magnetization is formally given by the Hellmann-Feynman equations as: \\ \\
\begin{equation}
\mathbf{M}=f_{n}\sum\limits_{n} \langle \psi_{n}|-\frac{\partial H_{\rm{AE}}}{\partial \mathbf{B}}|\psi_{n}  \rangle
\end{equation}
where $f_{n}$ is the occupation of the eigenstate $n$ where the expectation values is taken on ground-state spinor $\psi_{n}$.
Eq. 2 provides a direct evaluation of orbital magnetization in finite systems. However, extending this formula to periodic systems poses a challenge due to the undefined nature of the position operator in such systems. To overcome this limitation, the equation has been reformulated in the Wannier representation, which allows for the computation of orbital magnetization in the thermodynamic limit using the Berry-phase formula:\\ \\
\begin{equation}
\mathbf{M}=-\frac{\alpha N_{c}}{2N_{k}} Im\sum\limits_{n,\mathbf{k}} f_{n,\mathbf{k}}  \langle \partial_{\mathbf{k}} u_{n,\mathbf{k}}|\times \left ( H_{\mathbf{k}}+\epsilon_{n,\mathbf{k}} -2\epsilon_{F} \right ) |  \partial_{\mathbf{k}} u_{n,\mathbf{k}} \rangle
\end{equation} 
where $H_{\mathbf{k}}$ is the crystal Hamiltonian with $\mathbf{B}=0$, $\epsilon_{n,\mathbf{k}}$ and $u_{n,\mathbf{k}}$ are its eigenvalues and eigenvectors, $\epsilon_{F}$ is the Fermi level, $N_{c}$ is the number of cells and $N_{k}$ is the number of $\mathbf{k}$ points. The $ |  \partial_{k} u_{n,\mathbf{k}} \rangle$ is $\mathbf{k}$ the derivative of the Bloch wave function for $n$ occupied states. 
In the context of an all-electron method, Eq. (3) is valid for both normal periodic insulators and metals, as well as Chern insulators with a non-null Chern invariant. In the norm-conserving pseudopotential framework, instead, where the core regions of all-electron wave functions are replaced by smoother pseudo waves (PS), the application of the gauge including projection augmented wave (GIPAW) transformation (see Ref. \cite{PhysRevB.63.245101}) is necessary to derive the pseudo GIPAW Hamiltonian. At zero order of the magnetic field this transformation yields the following terms:\\ \\
\begin{equation}
H_{\rm{GIPAW}}^{(0,0)}=\frac{1}{2}\mathbf{p}^{2}+V_{loc(\mathbf{r})}+\sum_{\mathbf{R}}^{}V_{\mathbf{R}}^{NL}
\end{equation} 
\begin{equation}
H_{\rm{GIPAW}}^{(1,0)}=\frac{g'}{8}\alpha ^{2}\left [ \mathbf{\sigma } \cdot (\nabla V_{loc}(\mathbf{r})\times \mathbf{p})+\sum_{\mathbf{R}}^{}F_{\mathbf{R}}^{NL} \right ]
\end{equation} 
where $V_{loc(\mathbf{r})} $ is the local Kohn-Sham potential and $ V_{\mathbf{R}}^{NL} $ is the nonlocal pseudopotential in the separable Kleinmann-Bylander (KB) form: \\ \\
\begin{equation}
V_{\mathbf{R}}^{NL}=\sum_{nm}^{}|\beta_{\mathbf{R},n}  \rangle v_{\mathbf{R},nm} \langle \beta_{\mathbf{R},m}|
\end{equation} 
while $F_{\mathbf{R}}^{NL}$ is the separable nonlocal operator accounting the so-called paramagnetic contribution of the atomic site $\mathbf{R}$, that given the set of GIPAW projectors $|\widetilde{\rho}_{\mathbf{R},n}  \rangle $ can be written as:\\
\begin{equation} 
F_{\mathbf{R}}^{NL}=\sum_{\mathbf{R},nm}^{}|\widetilde{\rho}_{\mathbf{R},n}  \rangle \sigma \cdot {f}_{\mathbf{R},nm} \langle \widetilde{\rho}_{\mathbf{R},m} |
\end{equation} 
The expression ${f}_{\mathbf{R},nm}$  refer to the paramagnetic GIPAW integral and is given by Eq. 10 of Ref.\cite{PhysRevLett.88.086403}. At the first order in the magnetic filed, the GIPAW transformation yields the two terms:\\ \\
\begin{equation}
H_{\rm{GIPAW}}^{(0,1)}=\frac{\alpha}{2}\mathbf{B}\cdot \left(\mathbf{L}+\sum_{\mathbf{R}}^{}\mathbf{R}\times \frac{1}{i}\left [ \mathbf{r}, {V}_\mathbf{R}^{NL} \right ]\right)
\end{equation} 
\begin{align}
H_{\rm{GIPAW}}^{(1,1)} &= \frac{g'}{16}\alpha^{3}\mathbf{B} \cdot \left(\mathbf{r}\times(\sigma\times \nabla V_{loc}) + \sum_{\mathbf{R}}^{}E_{\mathbf{R}}^{NL} \right. \nonumber \\
&\quad \left. + \sum_{\mathbf{R}}\mathbf{R}\times \frac{1}{i}\left[ \mathbf{r}, {F}_\mathbf{R}^{NL} \right] \right)
\end{align} 
where $E_{\mathbf{R}}^{NL}$ similar to $F_{\mathbf{R}}^{NL}$, is the separable nonlocal operator accounting the diamagnetic contribution which use the GIPAW integral ${e}_{\mathbf{R},nm}$ given by Eq. 11 of Ref.\cite{PhysRevLett.88.086403}: 
\begin{equation}
E_{\mathbf{R}}^{NL}=\sum_{\mathbf{R},nm}^{}|\widetilde{\rho}_{\mathbf{R},n}  \rangle \cdot {e}_{\mathbf{R},nm} \langle \widetilde{\rho}_{\mathbf{R},m}|
\end{equation} 
Inserting the Hamiltonian first order terms, $H_{\rm{GIPAW}}^{(0,1)}$+$H_{\rm{GIPAW}}^{(1,1)}$, into Eq. (2) we obtain the final expression of orbital magnetization in the presence of nonlocal pseudopotential: 
\begin{equation}
\mathbf{M} = \mathbf{M}_{bare}+\Delta \mathbf{M}_{NL}+ \Delta \mathbf{M}_{para}+ \Delta \mathbf{M}_{dia}
\end{equation} 
\begin{equation}
\mathbf{M}_{bare}=\frac{\alpha}{2}\sum_{\mathbf{R}}^{}\langle\mathbf{r}\times\frac{1}{i} \left[\mathbf{r},  H_{GIPAW}^{(0,0)}+H_{GIPAW}^{(1,0)}  \right]      \rangle
\end{equation} 
\begin{equation}
\Delta \mathbf{M}_{NL}=\frac{\alpha}{2}\sum_{\mathbf{R}}^{}\langle{(\mathbf{R-r})}\times\frac{1}{i} \left[\mathbf{r-R}, {V}_\mathbf{R}^{NL}   \right]      \rangle
\end{equation} 
\begin{equation}
\Delta \mathbf{M}_{para}=\frac{g'\alpha^3}{16}\sum_{\mathbf{R}}^{}\langle{(\mathbf{R-r})}\times\frac{1}{i} \left[\mathbf{r-R}, {F}_\mathbf{R}^{NL}   \right]      \rangle
\end{equation} 
\begin{equation}
\Delta \mathbf{M}_{dia}=\frac{g'\alpha^3}{16}\sum_{\mathbf{R}}^{}\langle{} {E}_\mathbf{R}^{NL}       \rangle
\end{equation} 
where $\langle{}...\rangle$ stands for $\sum\limits_{n,\mathbf{k}} f_{n,\mathbf{k}}\langle \psi_{n}|...|\psi_{n}\rangle$. The $\mathbf{M}_{bare}$ can be easily computed by the modern theory of orbital magnetization of Eq. (3) where $H_{k}$ is the is the GIPAW Hamiltonian, and $\epsilon_{n,k}$ and $u_{n,k}$ are its eigenvalues and eigenvectors. The $\Delta \mathbf{M}_{NL}$, $\Delta \mathbf{M}_{para}$ and $\Delta \mathbf{M}_{dia}$ are respectively the nonlocal, paramagnetic and diamagnetic correction and their expression in the framework of Bloch functions can be found in Appendix B of Ref.\cite{PhysRevB.81.184424}. In the converse approach, for the sake of simplicity, the orbital magnetization is computed using a collinear approach where the total spin is aligned along an "easy axis" denoted as $\mathbf{e}$, which is typically one of three crystallographic directions. The choice of $\mathbf{e}$ affects the spin-orbit coupling within the system. Since the spin-orbit coupling is dependent on the orientation of the spin axis $\mathbf{e}$, the computed orbital magnetization is a function of $\mathbf{e}$. Therefore, to obtain a comprehensive understanding of the orbital magnetization within the system, it is necessary to compute the orbital magnetization as a function of the three different orientations of $\mathbf{e}$.
Finally, from the orbital magnetization the deviation of the g tensor $\Delta g_{\mu\nu}$ from the free electron values ($g_{e}=2.002 \ 319$) is obtained by the variation in  $\mathbf{M}$ with a spin flip: 
\begin{equation}
\Delta g_{\mu\nu}=-\frac{2}{\alpha}\mathbf{e_{\mu}}\cdot \frac{\mathbf{M}(\mathbf{e_{\nu}})-\mathbf{M}(\mathbf{-e_{\nu}})}{S-(-S)}=-\frac{2}{\alpha S} \mathbf{e_{\mu}}\cdot \mathbf{M}(\mathbf{e_{\nu}}) 
\end{equation} 
where $\mu\nu$ are Cartesian directions of the magnetic field, and $S$ is the total spin. 
\subparagraph{2.2 Converse NMR}
\vskip0.5cm
\noindent 
The formulation of the converse method for the calculation of NMR chemical shielding is very similar to the formulation presented above for the calculation of EPR, except that an additional vector potential term corresponding to a magnetic dipole $\mathbf{m}_{s}$ centered at the atom $\mathbf{s}$ and coordinates $\mathbf{r}_{s}$ is included: \\ \\
\begin{equation}
A_{s}(\mathbf{r})=\frac{\mathbf{m}_{s}\times(\mathbf{r}-\mathbf{r}_{s})}{|\mathbf{r}-\mathbf{r}_{s}|^3}
\end{equation} 
The AE Hamiltonian now become:
\begin{equation}
H_{\rm{AE}}=\frac{1}{2}\left\{\mathbf{p}+\alpha\left[ A(\mathbf{r})+A_{s}(\mathbf{r})\right]\right\}^{2}+V(\mathbf{r})
\end{equation} 
Applying the GIPAW transformation to the latter, similarly to what was done for the EPR, at zeroth order in the external magnetic field, we obtain a term $H_{\rm{GIPAW}}^{(0,0)}$ that is equal to Eq. (4), while the term $H_{\rm{GIPAW}}^{(1,0)}$ become:
\begin{equation}
H_{\rm{GIPAW}}^{(1,0)}=\frac{\alpha}{2} \left [  \mathbf{p}\cdot A_{s}(\mathbf{r}) +A_{s}(\mathbf{r})\cdot\mathbf{p}\right ]+\sum_{\mathbf{R}}^{}K_{\mathbf{R}}^{NL} 
\end{equation} 
where $K_{\mathbf{R}}^{NL}$ addressing the paramagnetic contribution of the magnetic dipole and it has the form of a nonlocal operator: \\ \\
\begin{equation}
K_{\mathbf{R}}^{NL}=\frac{\alpha}{2} \sum_{\mathbf{R},nm}^{}|\widetilde{\rho}_{\mathbf{R},n}  \rangle  \mathbf {k}_{\mathbf{R},nm} \langle \widetilde{\rho}_{\mathbf{R},m} |
\end{equation} 
where ${k}_{\mathbf{R},nm}$ refers to the GIPAW paramagnetic integrals that take into account the effect of the additional vector potential. These integrals, obtained from a set of AE partial waves $|\phi_{\mathbf{R},n}\rangle$ and pseudopotential partial waves $|\widetilde{\phi}_{\mathbf{R},n}\rangle$, can be written as:
\begin{align}
k_{\mathbf{R},nm} &=  \langle\phi_{\mathbf{R},n}|\mathbf{p}\cdot A_{s}(\mathbf{r}) +A_{s}(\mathbf{r})\cdot \mathbf{p}|\phi_{\mathbf{R},n}\rangle \nonumber \\
&\quad -\langle\widetilde{\phi}_{\mathbf{R},n}|\mathbf{p}\cdot A_{s}(\mathbf{r}) +A_{s}(\mathbf{r})\cdot \mathbf{p} |\widetilde{\phi}_{\mathbf{R},n}\rangle
\end{align} 
At first order in the external magnetic field, we obtain a term $H_{\rm{GIPAW}}^{(0,1)}$ that is equal to Eq. (8), while the term $H_{\rm{GIPAW}}^{(1,1)}$ become: 
\begin{align}
H_{\rm GIPAW}^{(1,1)} &= \frac{\alpha}{2}\mathbf{B} \cdot \Bigg(\mathbf{r}\times A_{s}(\mathbf{r}) 
  + \sum_{\mathbf{R}} J_{\mathbf{R}}^{\rm NL} \nonumber \\
  &\quad + \sum_{\mathbf{R}} \mathbf{R} \times \frac{1}{i}\left[ \mathbf{r}, {K}_\mathbf{R}^{\rm NL} \right] \Bigg)
\end{align} 
where $J_{\mathbf{R}}^{\rm NL}$ is the nonlocal operator: 
\begin{equation}
J_{\mathbf{R}}^{NL}=\sum_{\mathbf{R},nm}^{}|\widetilde{\rho}_{\mathbf{R},n}  \rangle  \mathbf {j}_{\mathbf{R},nm} \langle \widetilde{\rho}_{\mathbf{R},m} |
\end{equation} 
and the $\mathbf {j}_{\mathbf{R},nm}$ its GIPAW integrals addressing the diamagnetic contribution:
\begin{align}
j_{\mathbf{R},nm} &=  \langle\phi_{\mathbf{R},n}|(\mathbf{r-R})\times A_{s}(\mathbf{r}) |\phi_{\mathbf{R},n}\rangle \nonumber \\
&\quad - \langle\widetilde{\phi}_{\mathbf{R},n}|(\mathbf{r-R})\times A_{s}(\mathbf{r}) |\widetilde{\phi}_{\mathbf{R},n}\rangle
\end{align} 
As in the case of EPR, solving the Hellmann-Feynman equation using the first-order in magnetic field Hamiltonian terms, yields the orbital magnetization as the sum of $\mathbf{M}_{bare}$, $\Delta \mathbf{M}_{NL}$, $\Delta \mathbf{M}_{para}$ and $\Delta \mathbf{M}_{dia}$ terms like the Eq. 11. However, in the calculation of NMR, the $\Delta \mathbf{M}_{para}$ and $\Delta \mathbf{M}_{dia}$ terms are defined by replacing the nonlocal operators $E_{\mathbf{R}}^{NL}$ and $F_{\mathbf{R}}^{NL}$, respectively, with $K_{\mathbf{R}}^{NL}$ and $J_{\mathbf{R}}^{NL}$: 
\begin{equation}
\Delta \mathbf{M}_{para}=\frac{\alpha}{2}\sum_{\mathbf{R}}^{}\langle{(\mathbf{R-r})}\times\frac{1}{i} \left[\mathbf{r-R}, {K}_\mathbf{R}^{NL}   \right]      \rangle
\end{equation}  
\begin{equation}
\Delta \mathbf{M}_{dia}=\frac{\alpha}{2}\sum_{\mathbf{R}}^{}\langle{} {J}_\mathbf{R}^{NL}       \rangle
\end{equation} 
Finally, the chemical shielding tensor $\sigma_{s,\alpha\beta}\ $ is obtained form the derivative of the orbital magnetization $\mathbf{M}$ with respect to a magnetic point dipole $\mathbf{m}_{s} $, placed at the site of atom $s$ :
\begin{equation}
\sigma_{s,\alpha\beta}\ = \delta_{\alpha\beta}-\Omega\frac{\partial \mathbf{M}_{\beta}}{\partial \mathbf{m}_{s}}
\end{equation} 
where $\delta_{\alpha\beta}$ is the Kronecker delta and $\Omega$ is the volume of the simulation cell.
\vskip0.5cm
%%Implementation
\paragraph{3. Implementation} \
\vskip0.5cm
\noindent
In the {\fontfamily{qcr}\selectfont
QE-CONVERSE} code to compute the EPR g tensor or the NMR chemical shift, a special-purpose norm-conserving pseudopotential with GIPAW reconstruction is employed. These pseudopotentials, apart from the standard norm-conserving pseudopotentials, incorporate a complete set of AE core wavefunctions as well as AE ($\phi_{\mathbf{R},n}$) and PS ($\widetilde{\phi}_{\mathbf{R},n}$) partial waves. Both AE and PS partial waves are utilized to calculate the GIPAW integrals (${f}_{\mathbf{R},nm}$, ${e}_{\mathbf{R},nm}$, ${k}_{\mathbf{R},nm}$ and ${j}_{\mathbf{R},nm}$) and the GIPAW projectors ($\widetilde{\rho}_{\mathbf{R},n}$) at the beginning of the calculations. In the reference \cite{PhysRevB.81.184424}, detailed information on how norm-conserving pseudopotentials with GIPAW reconstruction are constructed are provided. A non-exhaustive library of GIPAW pseudopotentials can be found in Ref.\cite{pseudoGIPAW}.
In principle, the calculation of EPR g tensor or NMR shielding involves incorporating the SO coupling term for EPR, or the vector potential corresponding to the magnetic dipole in the case of NMR, into the Hamiltonian. The Kohn-Sham equation is then solved self-consistently under this Hamiltonian to converge to a new ground state. In the second step, the $\mathbf{k}$ derivative of the Bloch wave function is evaluated in order to determine the $\mathbf{M}_{bare}$ using the Eq. 3. In  {\fontfamily{qcr}\selectfont QE-CONVERSE} this derivative is obtained using a covariant finite difference formula, which corresponds to:
\begin{equation}
|\widetilde{\partial _{i}}u_{n \mathbf{k}}\rangle=\frac{1}{2}\left ( |\widetilde u_{n, \mathbf{k+q}}\rangle - |\widetilde u_{n, \mathbf{k-q}}\rangle  \right )    
\end{equation}
where is $|\widetilde u_{n, \mathbf{k+q}}\rangle$ the gauge-invariant "dual" state in mesh direction $i$ constructed as a linear combination of the occupied state $|u_{n, \mathbf{k+q}}\rangle$ at neighboring mesh point $\mathbf{q}$: \\ \\
\begin{equation}
|\widetilde{\partial _{i}}u_{n \mathbf{k}}\rangle=\sum_{n'}^{}\left ( S_{\mathbf{k+q}}^{-1} \right )_{n'n} | u_{n', \mathbf{k+q}}\rangle    
\end{equation}
where $(S_{\mathbf{k+q}}^{} )_{nn'}$ is the overlap matrix defined as:\\ \\
\begin{equation} 
    (S_{\mathbf{k+q}}^{} )_{nn'}=\langle u_{n, \mathbf{k}}| u_{n', \mathbf{k+q}}\rangle
\end{equation}
The diagonalization of the Hamiltonian, undertaken by both the routines dedicated to SCF calculations and those executing the $\mathbf{k}$ derivative of the Bloch functions, employs the Davidson method, addressed by the {\fontfamily{qcr}\selectfont cegterg} routine of Quantum ESPRESSO (Fig.\ref{fig:figure1}). \\
The nonlocal, paramagnetic, and diamagnetic correction terms are then computed starting from the converged ground state and using the nonlocal operators $E_{\mathbf{R}}^{NL}$, $F_{\mathbf{R}}^{NL}$, and $K_{\mathbf{R}}^{NL}$, $J_{\mathbf{R}}^{NL}$ respectively for the EPR g tensor and for the NMR chemical shielding. All the magnetization terms are summed as specified by Eq. 11 to yield the total orbital magnetization $\mathbf{M}$. At the end, once $\mathbf{M}$ is obtained, the deviation $\Delta g_{\mu\nu}$ is calculated according to Eq. 16. Alternatively, in the case of NMR, it is derived with respect to the magnetic dipole according to Eq. 27.
\vskip0.5cm
%%%%%%%%%%
\begin{figure*}
\centering
\includegraphics{Figures/Semi_advanced_flowchart_for_numerical_solution_procedure_v2.pdf}
\caption{Schematic view of the process to calculate the orbital magnetization in {\fontfamily{qcr}\selectfont QE-CONVERSE} code. }
\label{fig:figure1}
\end{figure*}
%%%%%%%%%%%
\paragraph{4. Installation and usage} \
\vskip0.5cm
\noindent
The repository of our code is released at GitHub (
\url{https://github.com/SFioccola/QE-CONVERSE}).
A Fortran 90 compiler and the Quantum ESPRESSO package must be previously installed. To take advantage of the enhancements in linear algebra operations, the scaLAPACK package or the ELPA library are required.
To install the code, the source files and the {\fontfamily{qcr}\selectfont Makefile} must be copied in the {\fontfamily{qcr}\selectfont /PP/} directory of Quantum ESPRESSO and typing {\fontfamily{qcr}\selectfont \$ make} the executable binary {\fontfamily{qcr}\selectfont qe-converse.x} is compiled. To execute a converse calculation with the {\fontfamily{qcr}\selectfont QE-CONVERSE}, the first step is to perform a SCF calculation using the {\fontfamily{qcr}\selectfont PW} code of Quantum ESPRESSO on a geometrically optimized structure. The SCF calculation must be spin-polarized inserting the flag {\fontfamily{qcr}\selectfont nspin= 2} in the input for {\fontfamily{qcr}\selectfont PW} code. The $\mathbf{k}$-points grid mesh should be set up without applying symmetry operations to the crystal, and given that the Kohn-Sham Hamiltonian breaks time-reversal symmetry, it is necessary to assume that the $k$ points and -$k$ points are not equivalent. This grid construction can be easily implemented including the flags {\fontfamily{qcr}\selectfont nosym=.true.} and {\fontfamily{qcr}\selectfont noinv=.true.}in the input for the SCF calculation. In the same directory where the calculation with the {\fontfamily{qcr}\selectfont PW} program is executed, the {\fontfamily{qcr}\selectfont QE-CONVERSE} code is launched with {\fontfamily{qcr}\selectfont ./qe-converse.x}  using an input file (an example for EPR and NMR calculation is provided in Appendix A). It is essential that the 
{\fontfamily{qcr}\selectfont prefix} and {\fontfamily{qcr}\selectfont outdir} flags match those used in the SCF calculation with the {\fontfamily{qcr}\selectfont PW} program. This is because the {\fontfamily{qcr}\selectfont QE-CONVERSE} code reads essential data such as atomic coordinates, ground state function, pseudopotentials, and the $\mathbf{k}$-points grid from the XML file in the {\fontfamily{qcr}\selectfont outdir} directory of a preceding spin-polarized SCF calculation. The flag {\fontfamily{qcr}\selectfont q} in the input file refers to the small vector used for the covariant derivative. Since its value influences the accuracy of the derivative, it is suggested to keep its default value of 0.01. To compute the EPR g tensor, the flag {\fontfamily{qcr}\selectfont lambda\textunderscore so(1,..,3)} must be included in the input file to incorporate the SO term into the Hamiltonian. The integer value within the parentheses indicates the direction in which the orbital magnetization is calculated. When calculating the NMR chemical shift instead, it is necessary to introduce into input file the nuclear dipole moment flag, {\fontfamily{qcr}\selectfont m\textunderscore0(1,..,3)} and the index of the atom carrying the dipole moment using the designated flag {\fontfamily{qcr}\selectfont m\textunderscore0\textunderscore atom}.
\vskip1.5cm
%%
\paragraph{5. Applications} \
\vskip0.5cm
\noindent
\subparagraph{5.1 the EPR g tensor for substitutional Nitrogen in Silicon}\ \\ \\
The following section presents the EPR g tensor calculation for the substitutional Nitrogen defect in Silicon (N\textsubscript{Si}). This defect has recently garnered significant theoretical and experimental attention due to its potential role as a spin donor in silicon-based quantum devices \cite{PhysRevB.89.115207,nano14010021}. It is characterized by the presence of two stable states, where the Nitrogen atom transitions from the tetra-coordinated on-center (T\textsubscript{d}) configuration site to a tri-coordinated off-center configuration (with C\textsubscript{3v} symmetry) along one of the ⟨111⟩ crystal directions, through a Jahn-Teller mechanism \cite{nano13142123}. In this example, we focus solely on the off-center configuration (Figure 2), referred as SL5 in experimental EPR signals. Regarding the g tensor in SL5, g\textsubscript{1} corresponds to the component aligned with the principal axis parallel to the ⟨111⟩ crystal direction (or g\textsubscript{$\parallel$}), while g\textsubscript{2} and g\textsubscript{3} represent the isotropic perpendicular components to the ⟨111⟩ crystal direction (or g\textsubscript{$\perp$}).\\
\begin{figure*}
\centering
\includegraphics[scale=0.4]{Figures/figure2.pdf}
\caption{Graphic representation of the off-center configuration for (N\textsubscript{Si}) defect. }
\label{fig:figure2}
\end{figure*}
%%
All ab-initio calculations were performed using the Perdew-Burke-Ernzerhof (PBE) exchange-correlation functional \cite{PhysRevLett.77.3865}, together with norm-conserving Trouiller-Martins pseudopotentials with GIPAW reconstruction and employing a plane wave basis set with a kinetic energy cutoff of 84 Ry. To model the (N\textsubscript{Si}) off-center point defect we used a 512 atoms supercells in which a substitutional nitrogen atom is embedded within a silicon cell. Geometry optimization was conducted by sampling at the $\Gamma$ point and setting the ionic convergence threshold to 0.026 eV/\r{A}. In the subsequent SCF and EPR g tensor calculation, we ensured convergence of the parameters by sampling $\mathbf{k}$-points on a $3\times 3\times 3$ mesh grid without considering symmetry, resulting in the integration of the Brillouin zone over a total of 54 $\mathbf{k}$-points. The convergence threshold was set to $1\cdot 10^{-8}$ Ry.\\
\vskip0.5cm

% Please add the following required packages to your document preamble:
% \usepackage{multirow}
\begin{table}[]
\setlength{\tabcolsep}{10pt}
\caption{The EPR $g$ tensor for the substitutional nitrogen 
point defect in silicon (N$_{\rm Si}$) calculated in this 
work using the {\fontfamily{qcr}\selectfont QE-CONVERSE } code compared to the values of  Ref. \cite{nano13142123},
in which a linear response approach is implemented. Results are compared 
to experimental data of Ref. \cite{PhysRevB.89.115207}.}
\vskip0.3cm
\label{table:epr}
\begin{center}
\begin{tabular}{llll}
\hline \hline
 Ref.   & $g_{1}$ &  $g_{2}$ & $g_{2}$ \\ 
\hline
 This work & 2.00209     & 2.00860     & 2.00860     \\ 
 Ref. \cite{nano13142123}      & 2.00200     & 2.00890     & 2.00780     \\ 
 Exp. \cite{PhysRevB.89.115207}      & 2.00219     & 2.00847     & 2.00847     \\ 
 \hline \hline
\end{tabular}
\end{center}
\end{table}
\noindent
The Table \ref{table:epr} presents the values of the g tensor calculated using the {\fontfamily{qcr}\selectfont QE-CONVERSE} code, compared with experimental results \cite{PhysRevB.89.115207} and the g tensor calculated by Simha et al.\cite{nano14010021} where the linear response method was implemented \cite{PhysRevLett.88.086403}. Our results demonstrate good agreement with the experimental data. Moreover, our calculations yield a g tensor that is fully consistent with the axial symmetry along the ⟨111⟩ crystal directions, as predicted by experiments, thus improving upon the results previously computed by the linear response method of Ref.\cite{nano14010021}.\\
%%
\subparagraph{5.2 \(\textbf{${}^{27}$Al}\) NMR chemical shifts in alumina}\ \\ \\
Corundum (\(\alpha\)-Al\(_2\)O\(_3\)) represents the most thermodynamically stable crystalline form of alumina and it can be synthesized through the calcination process starting from unactivated gibbsite (AlOH)\(_3\), which transforms into corundum via \(\gamma\) and \(\theta\)-Al\(_2\)O\(_3\) at temperatures exceeding 400°C. Determining the structure and the temperature at which transitions phase occur is crucial and it has applications in various industrial fields. For example, \(\gamma\)-Al\(_2\)O\(_3\) is used as a catalyst in the petrochemical industry. The most useful technique for this purpose is solid-state NMR of ${}^{27}$Al which detailed interpretation of the experimental spectra remains, however, a challenge. In this study, we apply the {\fontfamily{qcr}\selectfont QE-CONVERSE} code to perform ab-initio calculations of the NMR chemical shift of \ ${}^{27}$Al\ in \(\theta\) phase of alumina (Figure 3). In this phase, the experimental spectra of ${}^{27}$Al\ exhibit two distinct signals: one corresponding to the aluminum atoms octahedrally coordinated (Al\textsubscript{oct}), and the other to those tetrahedrally coordinated (Al\textsubscript{tet}) by oxygen atoms \cite{ODELL2007169}. Implementing the Eq. 27 we computed the absolute chemical shift tensors of \ ${}^{27}$Al\ nuclei that we converted into isotropic chemical shieldings using $\sigma_{iso}=$Tr$\left [ \sigma_{s}/3 \right ]$. Finally, we compared this last one with experimental isotropic chemical shifts by using the expression: $\delta _{iso}=\sigma _{ref}-\sigma _{iso}$, where $\sigma _{ref}$ in this work is the isotropic shielding of \ ${}^{27}$Al\ in \(\alpha\) phase, used as reference. \\ We performed DFT calculations with the PBE generalized gradient approximation and using the norm-conserving Trouiller-Martins pseudopotentials with GIPAW reconstruction. We used a cutoff energy of 70 Ry. We first relaxed a 200 atoms \(\theta\)-Al\(_2\)O\(_3\) supercell sampling at the $\Gamma$ point and setting the ionic convergence threshold to 0.026 eV/\r{A}. We then computed the SCF and NMR calculation setting the $\mathbf{k}$-points on a $2\times 2\times 2$ mesh grid without symmetric constraint. We converged the groud-state to $1\cdot 10^{-9}$ Ry. In Table 2 below, we present the chemical shifts of \ ${}^{27}$Al\ obtained in this work, compared to experimetal \cite{ODELL2007169} and theoretical data from Ref. \cite{PhysRevB.84.235119}, where a linear response method was implemented. Our results show very close agreement with both the experimental and previous theoretical results.
\begin{figure}[!htbp]
  \centering
  \includegraphics[width=\linewidth]{Figures/Al2O3.tif.pdf}
  \caption{Unit cell of \(\theta\)-alumina. \textbf{Al1} refers to the octahedrally coordinated aluminum atoms (Al\textsubscript{oct}) while \textbf{Al0} are the tetrahedrally coordinated (Al\textsubscript{tet}). }
  \label{fig:etichetta}
\end{figure}
%%
%%%
\begin{table*}
\end{table*}
\begin{threeparttable}
\caption{Comparison of chemical shifts calculated in this work and experimental and theoretical data from the literature.}
\setlength{\tabcolsep}{10pt} % Imposta lo spazio tra le colonne
\begin{tabular}{lllll}
\hline\hline
            & \multicolumn{4}{c}{\(\delta_{iso}\) (\(\mathrm{ppm}\))}                           \\ \cline{2-5} 
structure   & \multicolumn{2}{c}{previous} & \multicolumn{2}{c}{this work} \\ \cline{2-3}
Al site     & Exp.             & Th.       & \multicolumn{2}{l}{}          \\ \hline
\(\theta\)-Al\(_2\)O\(_3\) & \multicolumn{4}{c}{}                                         \\
Al\textsubscript{oct}       & -3.0 (1.0)\tnote{a}       & -4.9\tnote{b}      & \multicolumn{2}{c}{-6.3}      \\
Al\textsubscript{tet}       & 66.5 (1.0)\tnote{a}      & 62.6\tnote{b}     & \multicolumn{2}{c}{61.0}      \\ \hline\hline
\end{tabular}
\begin{tablenotes}
\item[a] Reference \cite{ODELL2007169}
\item[b] Reference \cite{PhysRevB.84.235119}
\end{tablenotes}
\label{tab:my-table}
\end{threeparttable}
%%%

\vskip 0.5cm
\paragraph{\textbf{Acknowledgments}} \ 
\vskip 0.5cm
\noindent
All the calculations were performed using the HPC resources of CALMIP Mesocenter (Toulouse) with grant P1555.
\vskip 0.5cm
\noindent

\vskip 0.5cm
\paragraph{\textbf{Appendix A}} \ 
\vskip 0.5cm
\noindent
Example of input for EPR g tensor calculation. {\fontfamily{qcr}\selectfont
\${DIR}} is an integer from 1 to 3 that denotes the spin-orbit coupling direction. 
\begin{verbatim}
&input_qeconverse
    prefix = 'example_EPR'
    outdir = './scratch/'
    diagonalization = 'david'
    verbosity = 'high'
    q_gipaw = 0.01
    dudk_method = 'covariant'
    diago_thr_init = 1d-4
    conv_threshold = 1e-8
    mixing_beta = 0.5
    lambda_so(${DIR}) = 1.0
/
\end{verbatim}
\vskip 0.5cm
Example of input for NMR calculation. {\fontfamily{qcr}\selectfont
\${DIR}} is an integer from 1 to 3 that referes to nuclear dipole moment direction while {\fontfamily{qcr}\selectfont
\${i}} indicates the atoms that brings the dipole.
\begin{verbatim}
&input_qeconverse
    prefix = 'example_NMR'
    outdir = './scratch/'
    diagonalization = 'david'
    verbosity = 'high'
    q_gipaw = 0.01
    dudk_method = 'covariant'
    diago_thr_init = 1d-4
    conv_threshold = 1e-8
    mixing_beta = 0.5
    m_0(${DIR}) = 1.0
    m_0_atom = ${i}
/
\end{verbatim}
\end{small}
%% main text


%% The Appendices part is started with the command \appendix;
%% appendix sections are then done as normal sections
%% \appendix

%% \section{}
%% \label{}

%% References
%%
%% Following citation commands can be used in the body text:
%% Usage of \cite is as follows:
%%   \cite{key}         ==>>  [#]
%%   \cite[chap. 2]{key} ==>> [#, chap. 2]
%%

%% References with bibTeX database:

\bibliographystyle{elsarticle-num}
\bibliography{sample}

%% Authors are advised to submit their bibtex database files. They are
%% requested to list a bibtex style file in the manuscript if they do
%% not want to use elsarticle-num.bst.

%% References without bibTeX database:

% \begin{thebibliography}{00}

%% \bibitem must have the following form:
%%   \bibitem{key}...
%%

% \bibitem{}

% \end{thebibliography}


\end{document}

%%
%% End of file 