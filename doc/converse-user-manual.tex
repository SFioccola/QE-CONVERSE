%%%%%%%%%%%%%%%%%%%%%%%%%%%%%%%%%%%%%%%%%%%%%%%%%%%%%%%%%%%%%%%%%%%%%%%
% QE-CONVERSE user's manual
%%%%%%%%%%%%%%%%%%%%%%%%%%%%%%%%%%%%%%%%%%%%%%%%%%%%%%%%%%%%%%%%%%%%%%%
\documentclass[a4paper,11pt,twoside]{article}
\usepackage[utf8]{inputenc}             % For accents and weird symbols
%\usepackage{braket}                     % bra-ket's
\usepackage{amsmath}                    % extra mathematical symbols
\usepackage{bm}                         % bold mathematical symbols
\usepackage{verbatim}                   % insert verbatim files
\usepackage[hang,small,bf]{caption}     % customizable captions
\usepackage{psfrag}                     % smart postscript substitutions
\usepackage{fancyvrb}                   % for including source code
\usepackage[in]{fullpage}               % use all page
\usepackage[pdftex]{graphicx}           % to insert figures
\usepackage{epstopdf}                   % convert eps to pdf
\DeclareGraphicsExtensions{.pdf,.png,.jpg,.eps}
\usepackage{hyperref}                   % crossref in PDF
\hypersetup{pdftex, colorlinks=true, linkcolor=blue, citecolor=blue,
            filecolor=blue, urlcolor=blue, pdftitle=, pdfauthor=Davide Ceresoli,
            pdfsubject=QE-GIPAW user's manual, pdfkeywords=}

%===================================================================================================
\title{QE-CONVERSE user's manual}
\author{Simone Fioccola, \texttt{sfioccola@laas.fr}}
\date{Last revision: \today}
%===================================================================================================

%===================================================================================================
\begin{document}
%===================================================================================================
\maketitle
\thispagestyle{empty}
%\tableofcontents
%\clearpage

%===================================================================================================
\section{Introduction}
%===================================================================================================
The QE-CONVERSE implement a non-perturbative approach (converse) to compute the orbital magnetization in isolated and periodic systems. The calculation of orbital magnetization allows ab-initio computation of macroscopic properties like the Nuclear Magnetic Resonance (NMR) chemical shifts and the Electronic Paramagnetic Resonance (EPR) g tensor.

%===================================================================================================
\section{Features}
%===================================================================================================
\begin{itemize}
  \item Periodic and isolated systems
  \item Norm-conserving pseudopotentials
  \item Parallelization over bands and g-vectors
  \item NMR shielding tensors 
  \item EPR g-tensor
  \item LDA and GGA functionals
\end{itemize}

%===================================================================================================
\section{Author contributions}
%===================================================================================================
{
  S. Fioccola, L. Giacomazzi, D. Ceresoli, N. Richard, A. Hemeryck, L. Martin-Samos
} 
%===================================================================================================
\section{Build instructions}
%===================================================================================================
This current version of the code is compatible with the version 7.2 of Quantum-Espresso package.
\begin{enumerate}
\item A Quantum-Espresso package version 7.2 must be previously installed (\url{https://gitlab.com/QEF/q-e/-/releases/qe-7.2}). To take advantage of the enhancements in linear algebra operations, the configuration with scaLAPACK package or ELPA library is suggested.
QE user's guide at \url{http://www.quantum-espresso.org/user_guide/user_guide.html}
\item Type:
\begin{verbatim}
    git clone https://github.com/mammasmias/QE-CONVERSE 
\end{verbatim}
This will download from github the latest stable version of QE-CONVERSE.
\item Type:
\begin{verbatim}
cd QE-CONVERSE
\end{verbatim}
\item 
 Copy the source files and the Makefile from \texttt{/src/} dictory into \texttt{/PP/} directory of Quantum-Espresso.
\item
 In the main directory of QE-7.2 type \texttt{make pp}. You should find the binary file \texttt{qe-converse.x} in the \texttt{/bin/} directory of Quantum-Espresso.
\end{enumerate}
%===================================================================================================
\section{Quick start}
%===================================================================================================
To calculate NMR/EPR parameters you need: 
\begin{enumerate}
\item pseudopotentials containing the GIPAW reconstruction (\url{https://sites.google.com/site/dceresoli/pseudopotentials})
\item run \texttt{pw.x} to perform the SCF calculation 
\item run \texttt{qe-converse.x} to calculate parameters (look into folder
\texttt{examples} for NMR shielding, EPR g-tensor.
)
\end{enumerate}

%===================================================================================================
\section{Input file description}
%===================================================================================================
The input file consists on only one namelist \texttt{\&input\textunderscore{}qeconverse} with the
following keywords:
\begin{description}

\item[prefix] (type: character, default: \texttt{'prefix'})\\
Description: prefix of files saved by program \texttt{pw.x}. The value of this keyword must be the same used in the SCF calculation.

\item[outdir] (type: character, default: \texttt{'./'})\\
Description: temporary directory for \texttt{pw.x} restart files. The value of this keyword must be the same used in the SCF calculation.

\item[diagonalization] (type: string, default: \texttt{'david'})\\
Description: diagonalization method (only allowed values: \texttt{'david'} )

\item[verbosity] (type: string default: \texttt{'high'})\\
Description: verbosity level (allowed values: \texttt{'low'}, \texttt{'medium'}, \texttt{'high'})

\item[q\_gipaw] (type: real, default: 0.01, units: bohrradius$^{-1}$)\\
Description: the small wave-vector for the covariant finite difference formula.

\item[dudk\_method] (type: string, default: \texttt{'covariant'})\\
Description: k-point derivative method (only allowed values: \texttt{'covariant'} )

\item[diag\_thr\_init] (type: real, default: 10$^{-7}$, units: Ry$^2$)\\
Description: Convergence threshold (ethr) for iterative diagonalization.

\item[conv\_threshold] (type: real, default: 10$^{-8}$, units: Ry$^2$)\\
Description: convergence threshold for the diagonalization in the SCF step.

\item[mixing\_beta] (type: real, default: 0.5)\\
Description: mixing factor for self-consistency.

\item[lambda\_so(1,..,3)] (type: real, default: 0.0, units: Bohr magneton)\\
Description: Cartesian components of electron spin. The value (1,..,3) denotes the spin-orbit coupling direction.

\item[m\_0(1,..,3)] (type: real, default: 0.0, units: nuclear magneton)\\
Description: Cartesian components of nuclear dipole. The value (1,..,3) denotes the nuclear dipole moment direction.

\item[m\_0\_atom] (type: integer, default: 0)\\
Description: Atom index carrying the nuclear magnetic dipole.


\end{description}


%===================================================================================================
\section{Limitations}
%===================================================================================================
Parallelization on k-point (pool) is not allowed. 


%===================================================================================================
\section{Resources}
%===================================================================================================
\begin{itemize}

\item NMR periodic table: \url{http://www.pascal-man.com/periodic-table/periodictable.html}

\end{itemize}


%===================================================================================================
\begin{thebibliography}{99}
%===================================================================================================
\bibitem{}
  

\end{thebibliography}

%===================================================================================================
\end{document}
%===================================================================================================

